\documentclass[norsk]{article}

\usepackage[norsk]{babel}
\usepackage{graphicx}
\usepackage{caption}
\DeclareCaptionFormat {empty}{}
\captionsetup{format=empty}
\usepackage[utf8]{inputenc}
\usepackage{placeins}
\usepackage{mathtools}
\usepackage{amsmath}
\usepackage{geometry}
\usepackage{capt-of}
\usepackage{listings}
\lstset{language=Python}
\lstset{frame=none}
\lstset{caption={Insert code directly in your document}}
\lstset{label={lst:code_direct}}
\lstset{basicstyle=\footnotesize}
%\renewcommand{\thesection}{}
%\renewcommand{\thesubsection}{}

\begin{document}
	\vspace*{1cm}
	\begin{center}
	{\huge FYS4150 - Project 1 \par}
	{\huge Linear 2nd order differential equations\par}
	\vspace{0.5cm}
	{\Large Dina ... \\ Kristine Garvin \\ Silje Christine Iversen \par}	
	\vspace{0.5cm}
	{\Large September 10, 2018 \par}
	\vspace{1cm}
	\end{center}
	
\section{Introduction}
In this project we will solve a general one-dimentional Poisson equation with Dirichlet boundary conditions by rewriting it as a set of linear equations. 

\section{Method}
We are to solve the equation $$-u"(x) = f(x),$$
with the Dirichlet boundary conditions \\ 
\begin{center}
$x\in (0,1), $ \hspace*{1cm} $u(0) = u(1) = 0.$
\end{center}
By defining a discretized approximation to $u$ as $v_i$, with grid points $x_i = ih$ in the interval from $x_0$ to $x_{n+1}=1$, we then have the boundary conditions $v_0=v_{n+1}=0$. Where the step lenght is defined as $h = 1/(n+1)$. We have approwimated the second derivative of u with a Taylor expansion: .................................................................................................................
$$\displaystyle-\frac{v_{i-1} - 2v_i + v_{i+1}}{h^2}=f_i\hspace*{1cm} 
for \hspace{3pt} i=1, 2,...,n$$
\\
\begin{align*}
\i=&1:   2v_1 - v_2 &= f_1/h^2 \\
\i=&2:  -v_1 + 2v_2 - v_3 &= f_2/h^2 \\ 
\i=&3:  -v_2 + 2v_3 - v_4 &= f_3/h^2 \\
\cdot \\
\cdot \\
\cdot \\
\i=&n:  -v_{n-1} + 2v_n &= f_n/h^2 \\ 
\end{align*}
\subsection{LU - decomposition}
\subsection{Gaussian elimination}
\section{Implementation}
\subsection{Numerical solution}
\subsection{Dynamical memory handling}
\section{Results}
\subsection{Analytic solution}
\begin{align*}
-\frac{d^2u(x)}{dx^2} &= f(x) \\[6pt]
u"(x) &= -100e^{-10x} \\[6pt]
u'(x) &= \int -100e^{-10x} dx\\[6pt]
u'(x) &= 10e^{-10x} + C \\[6pt]
u(x) &= \int \left(10e^{-10x} + C\right) dx \\[6pt]
u(x) &= -e^{-10x} + Cx + D \\
\text{Inserting boundary conditions}\\
u(0) &= -e^0 + D = 0 \\[6pt]
&\Rightarrow D = 1 \\[6pt]
u(1) &= -e^{-10} + C + 1 = 0 \\[6pt]
&\Rightarrow C = -(1 - e^{-10})\\
\\
u(x) &= 1 - (1 - e^{-10}) - e^{-10x}\\
\end{align*}
\subsection{Numerial solution}
\subsection{Errors}
\section{Conclution}

\section{References}

















\end{document}