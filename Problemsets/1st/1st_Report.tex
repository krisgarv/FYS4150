\documentclass[norsk]{article}


\usepackage{graphicx}
\usepackage{caption}
\DeclareCaptionFormat {empty}{}
\captionsetup{format=empty}
\usepackage[utf8]{inputenc}
\usepackage{placeins}
\usepackage{mathtools}
\usepackage{amsmath}
\usepackage{geometry}
\usepackage{capt-of}
\usepackage{listings}
\lstset{language=Python}
\lstset{frame=none}
\lstset{caption={Insert code directly in your document}}
\lstset{label={lst:code_direct}}
\lstset{basicstyle=\footnotesize}
%\renewcommand{\thesection}{}
%\renewcommand{\thesubsection}{}

\begin{document}
	\vspace*{1cm}
	\begin{center}
	{\huge FYS4150 - Project 1 \par}
	{\huge Linear 2nd order differential equations\par}
	\vspace{0.5cm}
	{\Large Dina Stabell \\ Kristine Garvin \\ Silje Christine Iversen \par}	
	\vspace{0.5cm}
	{\Large September 10, 2018 \par}
	\vspace{1cm}
	\end{center}
	
\section{Introduction}
In this project we will solve a general one-dimentional Poisson equation with Dirichlet boundary conditions by rewriting it as a set of linear equations. 

\section{Method}
The general one-dimentional Poisson equation with Dirichlet type boundary conditions:
\begin{align}
{-u_{xx}} &= f(x)
& \text{for}    
& x \in(0,1)
\end{align}
\begin{align}
{u(0)=u(1)=0}
\end{align}
When defining a finite difference approximation to this equation, we use Taylor expansion to approximate the derivative. Assuming u to be four-times continusly differentiable, the result is: 
\begin{align}
{u(x+h)=u(x)+hu'(x)+\frac{h^2}{2}v''(x)+\frac{h^3}{6}v^{(3)}(x)+\frac{h^4}{24}v^{(4)}(x+h_1)} 
\end{align}
\begin{align}
{u(x-h)=u(x)-hu'(x)+\frac{h^2}{2}u''(x)-\frac{h^3}{6}v^{(3)}(x)+\frac{h^4}{24}v^{(4)}(x-h_2)}
\end{align} 
Adding these two equations and reorganize a bit, gives
\begin{align}
{u''=-\frac{u_{i-1}-2u_{i}+u_{i+1}}{h^2}}
\end{align}
Now, by partition the unit interval into a finite numbers of subintervals, with the given gridpoints $x_{i}=ih$ where $h=\frac{1}{n+1}$. Now defining v to be the solution of the discrete problem, defined only at the gridpoints, the finite difference approximation is given by
\begin{align}
{-\frac{v_{i-1}-2v_{i}+v_{i+1}}{h^2}} &= {f(x_{i})}
& i=1,2,...,n
\end{align}
\begin{align}
{v_{0}=v_{n+1}=0}
\end{align}
When solving the first terms of the finite differece scheem, the result is
\begin{align}
{i=1:} &
& {-v_0 + 2v_1 -v_2=d_1}
\end{align}
Know that $v_0=0$, so this term will dissapear.
\begin{align}
{i=2:} &
& {-v_1+2v_2-v_3=d_2}
\end{align}
\begin{align}
{i=3:} &
& {-v_2+2v_3-v_4=d_3}
\end{align}
The last term looks like this
\begin{align}
{i=n:} &
& {-v_{n-1}+2v_n-v_{n+1}=d_n}
\end{align} 
where $v_{n+1}=0$ and $d_i=h^2f_i$.
If we group the unknowns in a vector $\mathbf{v}$, we see that the equations can be written as a linear set of equations of the form
\begin{align}
{\mathbf{Av}=\mathbf{d}}
\end{align}

\subsection{LU - decomposition}
\subsection{Gaussian elimination}
The idea in this elimination is to get rid of the elements right above and below the diagonal.
The first step is to get rid of the elements right below the diagonal. This is the forward substitution step of the process. Use the first equation to eliminate the first unknown, $v_1$, in the second equation, then to use the new version of the second equation to eiminate the first unknown,$v_2$, in the third equation and so on. The last equation will then contain only one unknown,$v_n$. \newline
The next step is to get rid of the elements right above the diagonal. This step is the backward substitution step in the process. Starting at the bottom with the last equation, you find the value of $v_n$ and use it to find the value of $v_{n-1}$ from the second to last equation. Then you use this solution to fint the value of $v_{n-2}$ from the third to last equation and so on.  \newline
The Gaussian elimination will provide an algorithm to solve the equations. For this algorithm and the linear system, $\mathbf{Av}=\mathbf{d}$, to be equivalent it is important that the elements in the algorithm for the diagonal are nonezero. A way of ensuring this is to check wether the matrix A is diagonal dominant. \newline
\textbf{Definition} \newline
\begin{align}
{\mid b_1 \mid >\mid c_1,}&{\mid b_i\mid \geq \mid a_i\mid \mid c_i\mid}
\end{align} 
\begin{align}
{\text{for}}&
&{ k=2,3,...,n}
\end{align}
\section{Implementation}
\subsection{Numerical solution}
\subsection{Dynamical memory handling}
\section{Results}
\subsection{Analytic solution}
The analytic solution of the equation(\textbf{number of eq?})
\begin{align}
{u_{xx}=100e^{-10x}}
\end{align}
\begin{align}
{u_x=10e^{-10x}+C}
\end{align}
\begin{align}
{u=-e^{-10x}+Cx+D}
\end{align}
Now, incerting the boundary conditions will provide the value of C and D
\begin{align}
{u(0)=0:} & 
& {-1 + D=0}  
\end{align}
\begin{align}
{\Rightarrow D=1}
\end{align}
\begin{align}
{u(1)=0:} &
& {-e^{-10}+C+1}
\end{align}
\begin{align}
{\Rightarrow C=e^{-10}-1}
\end{align}
So the solution is
\begin{align}
{u(x)=1-(1-e^{-10})x-e^{-10x}}
\end{align}
Which is the solution given in the task.
\subsection{Numerial solution}
\subsection{Errors}
\section{Conclution}

\section{References}

















\end{document}